\section{Fähigkeiten}
\cvitem{Technologien}{Express.js • PostgreSQL • OpenAPI Spezifikation (OAS) • MongoDB • React.js • Next.js • Kaskadierende Stilblätter (CSS) • Koa.js • Prisma ORM • JavaScript • TypeScript • HTML • Mongoose ODM • Material-UI}
\cvitem{Werkzeuge}{Git • VSCode • PGAdmin • Studio3T}

\section{Berufserfahrung}
\cventry{2023}
  {Software-Ingenieurin}
  {CodeLink}
  {Remote}
  {}
  {
    CodeLink ist eine soziale Netzwerkplattform für Entwickler zum Treffen und Teilen ihrer Fähigkeiten.
    Mitarbeit in einem vierköpfigen Team und Führung der Entwicklung einer Umgebung, in der Benutzer sowohl im Frontend als auch im Backend Projekte erstellen und sich engagieren können. 
    Dies umfasste auch Tests mit Jest und Supertest, Sicherstellen der Robustheit aller Routen und Erstellen von Datenbank-Seedings für realistischere Testumgebungen.
    Das Team versuchte, eine Authentifizierung zu integrieren, die jedoch bis zum Projektende nicht abgeschlossen war.
    Darüber hinaus integrierte das Team durch Nutzung einer externen News API einen Newsfeed und bot Nutzern die neuesten Technologie-Updates. Zusätzlich wurde eine GitHub API integriert, mit der Benutzer ihre GitHub-Projekte mühelos in ihr CodeLink-Portfolio übernehmen können.
  }

\cventry{2023}
  {Software-Ingenieurin}
  {ticktrackapp}
  {Remote}
  {}
  {
    Ticktrack ist eine Full-Stack-Webanwendung für effektives Zeitmanagement mit der Pomodoro-Technik.
    Mit React und Material-UI entwickelte ich ein responsives Frontend. 
    Während der Designphase nutzte ich die OpenAPI-Spezifikation und erstellte Sequenzdiagramme.
    Neben der Gestaltung der Benutzeroberfläche sorgte ich für Datenintegrität durch die Architektur eines umfassenden Backends, einschließlich eines gut strukturierten Datenmodells und der Implementierung von Controllern und Endpunkten mit Koa, PostgreSQL und Sequelize.
  }

\cventry{2023}
  {Software-Ingenieurin}
  {MedusaChat}
  {Remote}
  {}
  {
    Medusa ist eine Web-App für sofortige, spontane und anonyme Diskussionen.
    Ich überführte die gesamte Anwendung von Javascript nach Typescript.
    Einführung und Leitung von Verbesserungen in Bezug auf Barrierefreiheit, hohe Bewertungen im Chrome Lighthouse Accessibility Audit. 
    Erhöhung der Testinfrastruktur um 100\%, Abdeckung der Hauptfunktionalitäten von Grund auf mit Jest und E2E-Tests mit Cypress. 
    Integration von Husky für Git-Hooks und Implementierung anonymer Benutzerauthentifizierung mit Googles Firebase.
  }

\cventry{2020-2023}{Karrierepause}{Gesundheit und Wohlbefinden}{}{}{}

\cventry{2019}
  {Wissenschaftliche Hilfsarbeiterin}
  {Wissenschaftszentrum Berlin für Sozialforschung (WZB)}
  {Berlin}
  {Teilzeit}
  {Aktive Beteiligung an Forschungsprojekten, die in Publikationen, Präsentationen und Vorlesungen mündeten und das kollektive Verständnis der politischen Implikationen und Dimensionen der Digitalisierung erweiterten.}

\cventry{2017 -- 2019}
  {Wissenschaftliche Mitarbeiterin}
  {Technische Universität Berlin}
  {Berlin}
  {Teilzeit}
  {Leitung einer interdisziplinären Vorlesungsreihe, finanziert vom Bundesministerium für Bildung und Forschung, die die Bereiche Philosophie und Informatik verknüpfte.}

\section{Bildung}
\cventry{}
  {Intensiver Software-Engineering-Bootcamp}
  {Codeworks}
  {Remote}
  {}
  {}

\cventry{}
  {Master of Arts, Wissenschafts- und Technikphilosophie}
  {Technische Universität}
  {Berlin}
  {\textit{1,3}}
  {Während meines Masterstudiums arbeitete ich eng mit Ingenieuren im Bereich des autonomen Fahrens zusammen und forschte an der Schnittstelle von Philosophie, Software-Design und maschinellem Lernen.}

\cventry{}
  {Bachelor of Arts, Philosophie}
  {Ludwig-Maximilians-Universität}
  {München}
  {\textit{1,9}}
  {Mein Studium war durch eine tiefe Auseinandersetzung mit der Ethik der Technologie geprägt.}

\section{Sprachen}
\cvdoubleitem{Deutsch}{Muttersprache}{Englisch}{Verhandlungssicher}
\cvdoubleitem{Französisch}{Grundkenntnisse}{Spanisch}{Grundkenntnisse}